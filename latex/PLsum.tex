\documentclass[11pt]{article}
\usepackage[T1]{fontenc}
\usepackage{bookmark}
\usepackage{tocloft}
\usepackage{graphicx}
\usepackage{subcaption}
\usepackage[a4paper, total={6in, 9in}]{geometry}
\graphicspath{{../tutorial/}{../db_schema}{../json}}

\title{\textbf{Benchmark baz danych}}
\author{Aleks Zieliński, Filip Kalinowski}
\date{\today}

\hypersetup{hidelinks}
\renewcommand\contentsname{\hfill\bfseries\Large Rozdziały \hfill}
\renewcommand{\cftaftertoctitle}{\hfill}
\DeclareCaptionFormat{custom}
{%
    \textbf{#1#2}\textit{\small #3}
}
\captionsetup{format=custom}
\renewcommand{\figurename}{Zdjęcie.}

\begin{document}
\maketitle

\begingroup
\tableofcontents
\endgroup

\newpage
\section{Informacje wstępne}
	\subsection{Załozenia projektu}
		\begin{itemize}
			\item Zaprojektować schemat dokumentów JSON dla wybranego tematu. Należy mieć min 2 typy dokumentów na osobę w projekcie. Minimum jeden typ musi mieć dokumenty zagnieżdżone
			\item  Opracowany model, wraz z propozycją najważniejszych/najbardziej rozbudowanych poleceń należy zatwierdzić u wykładowcy (ew. poprawić wg. sugestii)
			\item Dla każdego typu dokumentu wygenerować min 100 sensownych instancji. Jeden z typów powinien mieć minimum 500 instancji. Dane można generować automatami typu generatedata.com bądź napisać 4 skrypt w pythonie (dodatkowe punkty).
			\item  Tak przygotowane dokumenty należy zaimportować do:
			\begin{enumerate}
				\item Wybranej bazy NoSQL
				\item PostgreSQLa do kolumn typu JSONB
				\item PostgreSQLa przy jednoczesnej konwersji JSON na tabele
			\end{enumerate}
			\item Wszystkie 3 procesy importowania należy udokumentować screenshotami i zapisanymi wykorzystanymi poleceniami
			\item  Należy utworzyć min 4 zapytania na osobę realizujące najważniejsze problemy wyszukiwania w tworzonej bazie. 
			\item Każde z zapytań musi mieć 3 wersje dla 3 sytuacji z pkt 4. 
			\item Należy przeprowadzić eksperyment pomiaru czasu wykonania tych zapytań. Aby eksperyment był rzetelnie przeprowadzony, zarówno baza NoSQLowa, jak i PostgreSQL muszą operować w podobnych warunkach: albo obie bazy są postawione w osobnych dockerach, albo zainstalowane w tym samym OS, ale uruchamiane jedna na raz (proces instalacji musi być udokumentowany). Wyniki czasowe należy przedstawić w tabelce
		\end{itemize}
	\subsection{Użyte technologie i sprzęt}
		\begin{itemize}
			\item PostgreSQL - wersja 16.9
			\item MongoDB - wersja 8.0.10
			\item Laptop:
			\begin{itemize}
				\item System operacyjny - Linux Mint 22.1
				\item CPU - Intel i7-9750H
				\item GPU - Nvidia GTX 1660
				\item RAM - 16GB 2667 MHz		
			\end{itemize}
		\end{itemize}
	
\newpage
	\subsection{Schemat logiczny bazy danych}
		\begin{figure}[h]
			\includegraphics[width=\textwidth]{db-bench.png}
			\caption{Schemat logiczny zrobiony w serwisie dbdiagram.io}
		\end{figure}
	\subsection{Przykładowe dane z plików JSON}
		\begin{figure}[h]
			\includegraphics[width=\textwidth]{JSON_examples_1.png}
			\caption{Pliki JSON od lewej: Cities, Continents, Countries, Discoveries}
		\end{figure}

\newpage
		\begin{figure}[h]
			\includegraphics[width=\textwidth]{JSON_examples_2.png}
			\caption{Pliki JSON od lewej: Field\_of\_sciences, Scientists}
		\end{figure}
\section{Instalacja baz danych}
	\subsection{PostgreSQL}
		\begin{figure}[!h]
			\includegraphics[width=\textwidth]{1_instalacja_postgresa.png}
			\caption{Instalacja za pomocą apt}
		\end{figure}
		\begin{figure}
			\includegraphics[width=\textwidth]{2_sprawdzenie_postgresa_i_psql.png}
			\caption{Po lewej - komenda systemctl aby sprawdzić czy działa, po prawej - logowanie na user, uruchomienie psqgl i test działania}
		\end{figure}
		\begin{figure}
			\includegraphics[width=\textwidth]{3_instalacja_pgadmin_wiem_co_robie.png}
			\caption{Instalacja pgadmina z serwera ftp PostgreSQL}
		\end{figure}

\newpage
	\subsection{MongoDB}
		\begin{figure}[h]
			\includegraphics[width=\textwidth]{4_importowanie_klucza_gpg.png}
			\caption{Importowanie kluczy gpg do MongoDB}
		\end{figure}
		\begin{figure}[!h]
			\includegraphics[width=\textwidth]{5_tworzenie_pliku_z_listami_source_do_repo.png}
			\caption{Stworzenie pliku z listą źródeł do pobierania repo/package}
		\end{figure}
		\begin{figure}
			\includegraphics[width=\textwidth]{6_instalacja_mongodb.png}
			\caption{Instalacja MongoDB}
		\end{figure}
		\begin{figure}
			\includegraphics[width=\textwidth]{7_zmiana_ustawien_dotyczacych_plikow.png}
			\caption{Zwiększenie limitu otwartych plików z 1024 do 64000}
		\end{figure}
		\begin{figure}
			\includegraphics[width=\textwidth]{8_uruchomienie_sprawdzenie_i_aktywacja_autouruchamiania_mongodb.png}
			\caption{Uruchomienie MongoDB, sprawdzenie statusu i włączenie autouruchamiania}
		\end{figure}
\section{Benchmark}
	\subsection{Zapytania}
		\begin{enumerate}
			\item Wybierz państwo które ma najwięcej odkryć z fizyki oraz nazwa państwa kończy się na "a"
			\item Znajdź najstarsze i najmłodsze odkrycie z USA z fizyki
			\item Znajdź państwa które mają odkrycia z biologii po 1960 roku
			\item Znajdź średnią liczbę wszystkich odkryć na rok w każdym państwie w europie
			\item Wypisz dziedzine w której każde państwo z europy dokonało najwięcej odkryć
			\item Wypisz średnią liczbę odkryć na naukowca z każdego kontynetu
			\item Znajdź państwo z największą średnią liczbą odkryć na osobę
			\item Znajdź miasto z którego pochodzi najwięcej fizyków
		\end{enumerate}
	\subsection{Zapytania w PostgreSQL}
		mnkln
	\subsection{Zapytania w PostgreSQL w JSONB}
		faafefe
	\subsection{Zapytania w MongoDB}
		cxzkm
	\subsection{Zapytanie xyz}
		iohntr
	\subsection{Zapytanie abc}
		opnfj

\end{document}